
% LICENSE:
% Creative Commons Attribution-Sharealike 4.0 license (a.k.a. CC BY-SA) (#ccbysa).
% See the file LICENSE in this source distribution.

\documentclass[10pt,aspectratio=169]{beamer}
%% \documentclass[handout,10pt,aspectratio=169]{beamer}

\newcommand{\pitem}{\pause\item}

\newenvironment{colblock}[3]{%
  \setbeamercolor{block body}{#2}
  \setbeamercolor{block title}{#3}
  \begin{block}{#1}}{\end{block}}

%% Red Block with only Body
\newenvironment{AlternativeRBBlock}[0]{%
  \setbeamercolor{block body}{bg=red!20}
  \begin{block}{}}{\end{block}}

%% red block with only body
\newenvironment{rbblock}[0]{%
  \begin{colblock}{}{bg=red!20}{bg=green}}{\end{colblock}}


\setbeamersize{description width=0.35cm}

%% \usepackage[backend=biber,style=numeric-comp,sorting=none]{biblatex}
%% \addbibresource{2020-CS-PD.bib}

\setbeamertemplate{blocks}[rounded][shadow=true]
\setbeamercolor{block body alerted}{bg=alerted text.fg!10}
\setbeamercolor{block title alerted}{bg=alerted text.fg!20}
\setbeamercolor{block body}{bg=structure!10}
\setbeamercolor{block title}{bg=structure!20}
\setbeamercolor{block body example}{bg=green!10}
\setbeamercolor{block title example}{bg=green!20}

% add outline at the start of each new section

\usepackage{upquote,textcomp}   % get correct ascii quotes in verbatim
\usepackage{cclicenses}

% allow positioning of one block over another
\usepackage[absolute,overlay]{textpos}
% while debugging absolute positions, it can be useful to uncomment
% the next line
%% \usepackage[texcoord,grid,gridunit=mm,gridcolor=red!10,subgridcolor=green!10]{eso-pic}

\usepackage[24hr,iso]{datetime}
\renewcommand{\dateseparator}{-}

\usepackage{bookmark}
\usepackage{etoolbox}
\usepackage{relsize}

%% don't put those navigation buttons at the bottom of the slide -
%% they never really get used
\beamertemplatenavigationsymbolsempty

%% \logo{\includegraphics[height=1cm]{algorithm-design-manual.png}}

%% for source code listings
\usepackage[T1]{fontenc}
\usepackage{textcomp}
\usepackage{lmodern}
\usepackage{moresize}
\usepackage[procnames]{listings}
%% special instructions to make sure we can paste from pdf
\makeatletter
\def\addToLiterate#1{\edef\lst@literate{\unexpanded\expandafter{\lst@literate}\unexpanded{#1}}}
\lst@Key{add to literate}{}{\addToLiterate{#1}} \makeatother

%% customization of listings
\usepackage{color}

%% colors for various parts of the syntax
\definecolor{mygreen}{rgb}{0,0.6,0}
\definecolor{mygray}{rgb}{0.5,0.5,0.5}
\definecolor{mymauve}{rgb}{0.58,0,0.82}
\lstset{commentstyle=\color{mygreen}}
\lstset{keywordstyle=\color{blue}}
\lstset{rulecolor=\color{black}}
\lstset{stringstyle=\color{mymauve}}

\lstset{framexleftmargin=5mm, frame=shadowbox, rulesepcolor=\color{gray!25}}
\lstset{breaklines=true}
\lstset{columns=fullflexible}
\lstset{keepspaces=true}
\lstset{showstringspaces=false}
\lstset{showspaces=false}
\lstset{extendedchars=false}
\lstset{literate={-}{-}1}
\lstset{upquote=true}
\lstset{add to literate={~}{\ttil}{1}}

\usepackage{graphicx}
\usepackage{tikz}
\usepackage{hyperref}
\usepackage{url}
\usepackage{media9}
\usepackage{setspace}
\usepackage{smartdiagram}

\usetikzlibrary{calc,trees,positioning,arrows,chains,shapes.geometric,decorations.pathreplacing,decorations.pathmorphing,shapes,matrix,shapes.symbols}

\title{Entropy in Evolutionary Algorithms}
\subtitle{Statistical mechanics bearing insight into evolution}
%\author{Lucas Blakeslee and Aengus McGuinness}
\subject{Optimization, Evolution, Genetic Algorithms}

%\institute[SF High]{
%  Santa Fe High School \\
%  Institute for Computing in Research}
\date{2021-03-24 \\
%  {\smaller[2] Last built \today{}T\currenttime } \\
  \smallskip
      {\smaller[4] (You may redistribute these slides with their \LaTeX\
        \vspace{-0.1cm}
        source code under the terms of the \\
        Creative Commons Attribution-ShareAlike 4.0 public license)}
}

\begin{document}

\section*{Frontmatter}

\begin{frame}
  \maketitle
\end{frame}

\section{Abstract}

\begin{frame}{Abstract}
	Genetic algorithms (GAs) are an optimization technique inspired by
	natural selection. GAs have yielded good results in certain practical
	problems, yet there is still more to be understood about their
	behavior on a theoretical level. One approach is to look at the
	evolutionary process from the point of view of statistical mechanics,
	and interpreting jumps in fitness as phase transitions. Toward this
	goal we examine the behavior of \emph{entropy} in a GA that optimizes
	a simple function.  We find that entropy increases as a new species
	diversifies, but its upper bound decreases with most phase
	transitions (which correspond to evolutionary steps).
\end{frame}

\begin{frame}{Introduction}
	\begin{itemize}
		\item Genetic Algorithms (GAs) are stochastic search algorithms that take inspiration from evolutionary processes. 
		\item Entropy is a generic term that means disorder. You can calculate it many different ways one method is shannonian information.
		\item We were motivated to do in depth analysis of genetic algorithms in order to understand their behavior better on a theoretical level.
		\item Schrödinger's book, "What is Life", sparked a modern wave of interest in the study of thermodynamics and evolution together.
	\end{itemize}
\end{frame}

\begin{frame}{Research Question}
		What happpens to entropy as a genetic algorithm
                evolves over time?
\end{frame}

\begin{frame}{Hypothesis}
	We hypothesized that entropy would gradually rise as small non-significant
	mutations occured, and that occasionally one highly beneficial mutation in
	an organism would cause entropy to rapidly decrease as that mutation was
	selected for and spreads throughout the population, and that entropy would then
	begin to rise again in a cyclical pattern. We hypothesized that if we're considering
	entropy to be the number of possible states an organism can be in, the
	upper bounds of the entropy would never reach levels they were previously at.
\end{frame}

\begin{frame}{Genetic Algorithms}
	In brief Genetic Algorithms attempt to find the maximum value
        of a function (called a fitness function).They do so by having
        a population of candidate values that take steps through the
        domain of that function.  The steps have a random component,
        but the randomness is directed by a principle akin to natural
        selection: when a population member has a high fitness value,
        it is more likely to survive and breed new population members
        that share some of its characteristics.
	\begin{center}
		\smartdiagram[flow diagram:horizontal]{Population,
                  Fitness Evaluation, Selection, Crossover, Mutation}
	\end{center}
\end{frame}

\begin{frame}{Punctuated Equilibrium}
      	\centering
        \includegraphics[width=0.75\textwidth]{Punctuated_Equilibrium.pdf}\\ \tiny
        Ian Alexander, CC BY-SA 4.0
        <https://qcreativecommons.org/licenses/by-sa/4.0>, via
        Wikimedia Commons
\end{frame}

\begin{frame}{Entropy}
		Shannon entropy:
$$
H({\rm P}) = -\sum_{x \in {\rm P}} p(x)\log p(x)
\label{eq:shannon-entropy-def}
$$
		\\
		\medskip
		\begin{itemize}
			\item "Amount of information" contained within
                          a variable.
			\item In this case, the number of possible states an organism can be in
			\item $p(x)$ is the occupancy of each state in the population
		\end{itemize}
		\bigskip
		See \emph{entropy\_simple\_example.py} for a simple
                demo of how Shannon entropy is calculated
\end{frame}

\part{Part I, day 1}

\section{Experimentation}

\begin{frame}{Our GA: goals}
  We designed the problem and the genetic algorithm with two goals in
  mind: one more immediate, the other one inspired by trying to gain a
  deeper understanding.
  \vspace{-0.2cm}
  \begin{columns}[t]
    \begin{column}{0.5\textwidth}
      \begin{block}{Immediate}
        \begin{itemize}
          \pause\item
          Devise a procedure to look for the maximum value of a
          function for which it can be hard to find that analytically.\\
          \pause\item
          Write the code somewhat generally so that it can be applied
          to other problems.\\
          \pause\item
          Write visualization software to explore the relationship
          between the solution and the \emph{fitness landscape}.\\
        \end{itemize}
      \end{block}
    \end{column}
    \begin{column}{0.5\textwidth}
      \pause
      \begin{block}{What is bubbling beneath?}
        \begin{itemize}
          \pause\item
          We have a population of thousands of members.  What
          describes the state of that population?
          \pause\item
          When is the population ``ripe'' for a jump in fitness?
          \pause\item 
          Can one learn about when to \emph{finish} a search?
        \end{itemize}
      \end{block}
    \end{column}
  \end{columns}
  \pause
  \begin{block}{The holy grail}
    GAs are an example of a \emph{stochastic search} technique,
    because they use random numbers to search a large space.  The holy
    grail of stochastic search is to know when you have a ``close
    enough'' solution so that you can halt the search.
  \end{block}
\end{frame}

\begin{frame}{Our GA: procedure}
  \begin{block}{Components}
  \begin{description}
  \item [Population of our GA] A collection of floating
    point numbers.  (Floating point numbers approximate numbers on the
    real line).\\
  \item [Chromosomes/DNA] The 32 bits which represent a
    floating point number in the IEEE 754 standard.  Each individual
    has/is a single chromosome.
  \item [Genes] The individual bits in a floating point numbers
  \end{description}
  \end{block}
  \pause
  \begin{block}{The evolution}
    \begin{description}
      \item [select by fitness] Evaluate the fitness function.
      \item [mate parents] Mate the fittest half of the population to
        produce children.
      \item [mutate children] Randomly change the DNA of children,
        either by flipping random bits in the floating point number,
        or by having them drift along the real line.
    \end{description}
  \end{block}
\end{frame}

\begin{frame}{Our GA: miscellenea}
  \begin{block}{Tuning}
          \begin{description}
          \item [type of mutation] Bit flip or drift.
          \item [scale of drift] Depending on how the fitness landscape
            oscillates, different drift sizes might fit that scale better.
        \item [population/generations] How many members are there, and
          for how many generations do we run?
        \end{description}
  \end{block}
\end{frame}


\begin{frame}{Fitness landscape}
  A visualization of the fitness function which
  shows the peaks in fitness, as well as the pitfalls in searching for
  the peaks.
  \begin{figure}
    \centering
    \resizebox{\linewidth}{!}{%
      \includegraphics{fit-func_pid3815286.pdf}
      \includegraphics{fit-func_pid3728854.pdf}
    }
    \caption{Two examples of fitness functions: one has a $\sin()$ wave
      riding on a gaussian, the other has a flat-top wave riding on it.
      Most of our results are based on the ``flat-top'' function.}
    \label{fig:fit-func_pid3095791}
  \end{figure}
\end{frame}

\begin{frame}{Results}
  \begin{columns} % This creates a frame with multiple columns.
    % The first column will be 50% as wide as the width of text on the page.
    \begin{column}{0.4\textwidth}
      % Beamer doesn't like to display .eps files. This .png was
      % converted from .eps using Adobe Acrobat. The file graph1.png
      % should be in the same folder as the .tex file.
      \includegraphics[width=\textwidth]{GA-gen-info_pid3095791.out.pdf}
                      {\tiny
                      	\begin{singlespace}
                      	 An example of evolution.  The top panel shows
                         fitness as a function of generation for a
                         ``$\sin()$ on top of gaussian'' fitness
                         landscape.  The middle panel shows entropy,
                         where you can see that the end point of the
                         evolution has a lower entropy than earlier
                         phases.
              			\end{singlespace}    
              }
    \end{column}

    \begin{column}{0.6\textwidth}
     \includegraphics[width=\textwidth]{GA-gen-info_pid3027915.out.pdf}\\ {\tiny
      				\begin{singlespace}
      				Another example of evolution that shows more clearly how
      				the entropy reaches lower pleateaus with increased fitness.
      				\end{singlespace}	
      		}
    \end{column}
  \end{columns}
\end{frame}

\begin{frame}{Discussion}
	Entropy rapidly falls as the average fitness makes large steps. This
	is because as the x-position value nears the local maxima the
	preceding values fall out of the population as the fittest individuals
	progress to the next generation. We observed occasional dramatic jumps
	in fitness as opposed to a linear development. This is corresponds
	with punctuated equilibrium, a theory in evolutionary biology that
	evolutionary development is marked by isolated episodes of rapid
	speciation between long periods of little or no change (Punctuated
	Equilibrium, 2020).
\end{frame}

\begin{frame}{Conclusion}
	From this paper we have concluded that:
	\begin{itemize}
		\item The results supported our hypothesis in multiple ways:
			\begin{itemize}
				\item Entropy and evolution are related
				\item Drops in entropy corresponded with increases in fitness
				\item The upper bounds of entropy never reached above its previous levels
			\end{itemize}
		
	\end{itemize}

\end{frame}

\begin{frame}{References}
Mitchell, Melanie. An introduction to genetic algorithms. MIT press, 1998.
\medskip
	Kinnear, K. E. (1994). In K. E. Kinnear (Ed.), \emph{Advances in
	Genetic Programming} (pp. 3-17). Cambridge: MIT Press.\\
	\medskip
	Punctuated Equilibrium. (2020). In Oxford Online Dictionary. Retrieved
	from $https://www.lexico.com/definition/punctuated\_equilibrium$\\
	\medskip
	Radcliffe N.J., Surry P.D. (1995) Fundamental limitations on search
	algorithms: Evolutionary computing in perspective.  In: van Leeuwen
	J. (eds) Computer Science Today. Lecture Notes in Computer Science,
	vol 1000. Springer, Berlin, Heidelberg.
	$https://doi.org/10.1007/BFb0015249$\\
	
\end{frame}

%% \input{curriculum}

%% \input{programming-languages}

%% \part{Part I, day 2}

%% \input{pedagogical}

%% \input{tools}

%% \input{bones-of-the-world}

%% %% \input{extras}

%% %% \input{references}

\end{document}
