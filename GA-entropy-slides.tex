% LICENSE:
% Creative Commons Attribution-Sharealike 4.0 license (a.k.a. CC BY-SA) (#ccbysa).
% See the file LICENSE in this source distribution.

\documentclass[10pt,aspectratio=169]{beamer}
%% \documentclass[handout,10pt,aspectratio=169]{beamer}

\newcommand{\pitem}{\pause\item}

\newenvironment{colblock}[3]{%
  \setbeamercolor{block body}{#2}
  \setbeamercolor{block title}{#3}
  \begin{block}{#1}}{\end{block}}

%% Red Block with only Body
\newenvironment{AlternativeRBBlock}[0]{%
  \setbeamercolor{block body}{bg=red!20}
  \begin{block}{}}{\end{block}}

%% red block with only body
\newenvironment{rbblock}[0]{%
  \begin{colblock}{}{bg=red!20}{bg=green}}{\end{colblock}}


\setbeamersize{description width=0.35cm}

%% \usepackage[backend=biber,style=numeric-comp,sorting=none]{biblatex}
%% \addbibresource{2020-CS-PD.bib}

\setbeamertemplate{blocks}[rounded][shadow=true]
\setbeamercolor{block body alerted}{bg=alerted text.fg!10}
\setbeamercolor{block title alerted}{bg=alerted text.fg!20}
\setbeamercolor{block body}{bg=structure!10}
\setbeamercolor{block title}{bg=structure!20}
\setbeamercolor{block body example}{bg=green!10}
\setbeamercolor{block title example}{bg=green!20}

% add outline at the start of each new section
\AtBeginSection[]{
 \begin{frame}{Outline}
   \tableofcontents[currentsection]
 \end{frame}
}

\AtBeginSubsection[]{
 \begin{frame}{Outline}
   \tableofcontents[currentsubsection]
 \end{frame}
}

\usepackage{upquote,textcomp}   % get correct ascii quotes in verbatim
\usepackage{cclicenses}

% allow positioning of one block over another
\usepackage[absolute,overlay]{textpos}
% while debugging absolute positions, it can be useful to uncomment
% the next line
%% \usepackage[texcoord,grid,gridunit=mm,gridcolor=red!10,subgridcolor=green!10]{eso-pic}

\usepackage[24hr,iso]{datetime}
\renewcommand{\dateseparator}{-}

\usepackage{bookmark}
\usepackage{etoolbox}
\usepackage{relsize}

%% don't put those navigation buttons at the bottom of the slide -
%% they never really get used
\beamertemplatenavigationsymbolsempty

%% \logo{\includegraphics[height=1cm]{algorithm-design-manual.png}}

%% for source code listings
\usepackage[T1]{fontenc}
\usepackage{textcomp}
\usepackage{lmodern}
\usepackage{moresize}
\usepackage[procnames]{listings}
%% special instructions to make sure we can paste from pdf
\makeatletter
\def\addToLiterate#1{\edef\lst@literate{\unexpanded\expandafter{\lst@literate}\unexpanded{#1}}}
\lst@Key{add to literate}{}{\addToLiterate{#1}}
\makeatother

%% customization of listings
\usepackage{color}

%% colors for various parts of the syntax
\definecolor{mygreen}{rgb}{0,0.6,0}
\definecolor{mygray}{rgb}{0.5,0.5,0.5}
\definecolor{mymauve}{rgb}{0.58,0,0.82}
\lstset{commentstyle=\color{mygreen}}
\lstset{keywordstyle=\color{blue}}
\lstset{rulecolor=\color{black}}
\lstset{stringstyle=\color{mymauve}}

\lstset{framexleftmargin=5mm, frame=shadowbox, rulesepcolor=\color{gray!25}}
\lstset{breaklines=true}
\lstset{columns=fullflexible}
\lstset{keepspaces=true}
\lstset{showstringspaces=false}
\lstset{showspaces=false}
\lstset{extendedchars=false}
\lstset{literate={-}{-}1}
\lstset{upquote=true}
\lstset{add to literate={~}{\ttil}{1}}

\usepackage{graphicx}
\usepackage{tikz}
\usepackage{hyperref}
\usepackage{url}
\usepackage{media9}


\title{Entropy in Evolutionary Algorithms}
\subtitle{Statistical mechanics bearing insight into evolution}
\author{Lucas Blakeslee and Aengus McGuinness}
\subject{Optimization, Evolution, Genetic Algorithms}

\institute[SF High]{
  Santa Fe High School \\
  Institute for Computing in Research}
\date{2021-03-20 \\
  {\smaller[2] Last built \today{}T\currenttime } \\
  \smallskip
      {\smaller[4] (You may redistribute these slides with their \LaTeX\
        \vspace{-0.1cm}
        source code under the terms of the \\
        Creative Commons Attribution-ShareAlike 4.0 public license)}
}

\begin{document}

\section*{Frontmatter}

\begin{frame}
  \maketitle
\end{frame}

\begin{frame}{Outline} 
  \tableofcontents
\end{frame}


\section{Goals}

\begin{frame}{Goals and path}
  In the educational industrial complex we are required to state our
  goals before we start.\\
  It might even be a good idea.

  \begin{columns}[t]
    \begin{column}{0.5\textwidth}
      \begin{block}{Goals}
        \begin{itemize}
          \pause\item
          Have a broad view of University curriculum, successes and
          limitations, state of industry.\\
          \pause\item
          Awareness of grand challenges \\in software engineering.\\
          \pause\item
          Awareness of current approaches \\ to address those challenges.\\
        \end{itemize}
      \end{block}
    \end{column}
    \begin{column}{0.5\textwidth}
      \pause
      \begin{block}{The meandering path}
        \begin{itemize}
          \pause\item
          My path is largely historical because of \\ my personal inclination
          to use history \\ to give perspective.\\
          \pause\item
          We need perspective so we are not tossed about by the short-term
          interest of industry.
        \end{itemize}
      \end{block}
    \end{column}
  \end{columns}
  \pause
  \begin{block}{Style}
    \begin{itemize}
    \item
      Slides are placeholders for me to then tell stories.  \\ \pause
      I hope you will talk and tell stories too. \\ \pause
    \item
      But also: I join the modern quest to give a seminar made
      entirely of xkcd slides.
    \end{itemize}
  \end{block}
\end{frame}

\part{Part I, day 1}

\section{Setting the stage}

\begin{frame}{A picture}
  \begin{columns} % This creates a frame with multiple columns.
    % The first column will be 50% as wide as the width of text on the page.
    \begin{column}{0.4\textwidth}
      % Beamer doesn't like to display .eps files. This .png was
      % converted from .eps using Adobe Acrobat. The file graph1.png
      % should be in the same folder as the .tex file.
      \includegraphics[width=\textwidth]{GA-gen-info_pid3095791.out.pdf}
                      {\tiny Margaret Hamilton, who led the MIT
                        team that wrote the Apollo on-board
                        software in the 1960s, is one of the
                        coiners of the term ``software
                        engineering''.\par}
    \end{column}
    
    \pause

    \begin{column}{0.6\textwidth}
      \includegraphics[width=0.75\textwidth]{GA-gen-info_pid3095791.out.pdf}
      \\
      \pause
      Most of the computer science department courses.\\
      \pause
      Less math.\\
      \pause
      Process and management classes.\\
      \pause
      ISO's ``Software Engineering Body of Knowledge'' (SWEBOK).
      
    \end{column}
  \end{columns}
\end{frame}



%% \input{curriculum}

%% \input{programming-languages}

%% \part{Part I, day 2}

%% \input{pedagogical}

%% \input{tools}

%% \input{bones-of-the-world}

%% %% \input{extras}

%% %% \input{references}

\end{document}
